\section{Understanding Cyclomatic Complexity}

Cyclomatic complexity\footnote{\href{https://en.wikipedia.org/wiki/Cyclomatic_complexity}{https://en.wikipedia.org/wiki/Cyclomatilc\_complexity}} is a measurement created from the control flow of an application.
It uses the number of edges and nodes in the control flow graph along with the number of connected components.
The common formula is M = E - N + 2P, where
\begin{itemize}
	\item E = the number of edges of the graph.
	\item N = the number of nodes of the graph.
	\item P = the number of connected components.
\end{itemize}
There are other formulas depending on if the graph is a strongly connected graph but for this scenario those do not matter.

The simplest understanding of how to calculate this metric is every time the function makes a choice the one gets added to complexity and a function has a base level complexity of one.
Below are two example functions that produce the same output and both a cyclomatic complexity of two.
But the structure of each function is different.

% Python code block
\begin{center}
	% \codefile{Sample Functions}{./understanding_cc.py}
\end{center}

The above examples show that readability may have no effect on the \cc and by that point, the style of a language or a team of programmers should not affect the overall metric.
What will affect the metric is how the tooling authors interpret the language control flow features.
For example, a switch statement with cases that fall through to the next case can be counted as one or the total number of cases that are passed through.

As the Kuadrant project uses multiple languages, we can only use the metric from a high-level, trending viewpoint.
Scores across multiple languages should not be compared with each other.

\subsection{Classifying Cyclomatic Complexity}
The metric itself is a number but as a single number, the metric is hard to reason about, so it is bucketed into categories.
While researching this work, I came across a number of different bucketing systems.
I will use the bucket classification found on \href{https://radon.readthedocs.io}{radon.readthedocs.io}, it gives a good range at both ends of the spectrum.
These classifications can be seen in \textbf{Figure \ref{tab:cc_scores}}.
This ranking will be used and from a high level you want scores in the ABC grouping and not many Fs.
\begin{figure}

	\begin{center}
		\begin{tabular}{|l|l|l|}
			\hline
			\textbf{CC Score} & \textbf{Rank} & \textbf{Risk}                           \\
			\hline
			1 - 5             & A             & Low - simple block                      \\
			\hline
			6 - 10            & B             & Low - well-structured and stable block  \\
			\hline
			11 - 20           & C             & Moderate - slightly complex block       \\
			\hline
			21 - 30           & D             & More than moderate - more complex block \\
			\hline
			31 - 40           & E             & High - complex block, alarming          \\
			\hline
			40+               & F             & Very high - error-prone, unstable block \\
			\hline
		\end{tabular}
	\end{center}
	\caption{Table grouping \cc scores.}
	\label{tab:cc_scores}
\end{figure}
